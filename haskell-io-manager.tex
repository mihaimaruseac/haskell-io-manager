\documentclass{beamer}

\usepackage{color}			 % highlight
\usepackage{alltt}			 % highlight
\usepackage{code/highlight}	 % highlight

\usepackage{hyperref}
\mode<presentation>

\title{A New Haskell IO Manager}
\author{Mihai Maruseac, ROSEdu\\mihai@rosedu.org}

\setbeamertemplate{frametitle continuation}[from second]
\setbeamertemplate{footline}[frame number]

\pgfdeclareimage[height=7cm]{pong}{img/pong}
\pgfdeclareimage[height=6cm]{mightynginx}{img/mightynginx}
\pgfdeclareimage[height=6cm]{pipes}{img/1m1bpipes}
\pgfdeclareimage[height=6cm]{scalability}{img/scalability}
\pgfdeclareimage[height=6cm]{delay}{img/threaddelay}
\pgfdeclareimage[height=4cm]{rts}{img/rts}
\pgfdeclareimage[height=3cm]{native}{img/native}
\pgfdeclareimage[height=3cm]{event}{img/event}
\pgfdeclareimage[height=3cm]{ppc}{img/ppc}
\pgfdeclareimage[height=3cm]{green}{img/green}
\pgfdeclareimage[height=6cm]{mail}{img/mail}
\pgfdeclareimage[height=6cm]{old}{img/old}
\pgfdeclareimage[height=6cm]{new}{img/new}
\pgfdeclareimage[height=7cm]{pipeline}{img/pipeline}

\begin{document}

\maketitle

\begin{frame}{What is Haskell?}
  \begin{itemize}
    \item functional programming language
    \item 1990, academic research committee
    \item common language for research
    \pause
    \item purely functional
    \item lazy
    \begin{itemize}
      \item reasoning about program
      \item barrier for non-functional features
    \end{itemize}
    \pause
    \item strongly typed
    \item type inference
  \end{itemize}
\end{frame}

\begin{frame}{Usage}
  \begin{itemize}
    \item spread over multiple domains
    \begin{itemize}
      \item Maths and sciences
      \item games
      \item graphics
      \item compilers
      \item networking
    \end{itemize}
    \item declarative style
  \end{itemize}
  \input{code/cs.hs}
\end{frame}

\begin{frame}{Why Haskell?}
  \begin{itemize}
    \item declarative style
    \pause
    \item correctness
    \begin{itemize}
      \item strong static typing
      \item pure functions
      \item special types for impure functions
    \end{itemize}
    \pause
    \item performance
    \begin{itemize}
      \item statically compiled
      \item alternative to \texttt{C++} / \texttt{Java}
    \end{itemize}
  \end{itemize}
\end{frame}

\begin{frame}{GHC}
  \begin{itemize}
    \item \texttt{The Glorious Glasgow Haskell Compilation System}
    \pause
    \item started as academic research funded by UK
    \begin{itemize}
      \item freely available, robust, portable compiler
      \item high-performance code
      \item research for better compilers (learn how programs behave)
      \item modular foundation for other researchers
    \end{itemize}
    \pause
    \item Microsoft Research
    \item other contributors
    \item Open Source model
    \pause
    \item recommended suite for Haskell development
    \item \texttt{ghc}, \texttt{ghci}
  \end{itemize}
\end{frame}

\begin{frame}{Higher Level Structure}
  \begin{itemize}
    \item compiler itself
    \begin{itemize}
      \item Haskell source code $\rightarrow$ executable \textbf{binary} code
    \end{itemize}
    \pause
    \item Boot Libraries
    \begin{itemize}
      \item needed by compiler
      \item bootstrapping
      \item low level functionality: \texttt{Int} in terms of primitive types, $\ldots$
      \item high level functionality: efficient data structures for compiler
      and RTS
    \end{itemize}
    \pause
    \item Runtime System (RTS)
    \begin{itemize}
      \item C library
      \item running compiled Haskell code
      \item garbage collection, thread scheduling, profiling, exception
      handling, $\ldots$
      \item linked into every compiled program
      \item responsible for GHC's key strengths: concurrency and parallelism
    \end{itemize}
  \end{itemize}
\end{frame}

\begin{frame}{Compiling Haskell Code}
  \centering
  \pgfuseimage{pipeline}
\end{frame}

\begin{frame}{The Runtime System}
  \begin{itemize}
    \item profiling
    \item memory management
    \item parallel, generational, garbage collector
    \item dynamic linkage, primitive operations
    \item thread management and scheduling
  \end{itemize}
  \centering
  \pgfuseimage{rts}
\end{frame}

\begin{frame}{The Runtime System (2)}
  \begin{itemize}
    \item bits to execute Haskell code which cannot be translated into binary
    \begin{itemize}
      \item code to throw exception
      \item code to allocate \texttt{\#Array} objects
      \item $\ldots$
    \end{itemize}
    \item STM support
    \item deterministic locking
  \end{itemize}
  \texttt{<exe>: thread blocked indefinitely in an MVar operation}
\end{frame}

\begin{frame}{Threading}
  \begin{itemize}
    \item concurrency = important abstraction
    \item make it cheap, programmers avoid costly abstractions/paradigms
    \item no thread-pools management by programmer
    \item just works
    \pause
    \item OSes struggle with 1000s of threads
    \item GHC wanted/capable to handle milions
    \pause
    \item models:
    \begin{itemize}
      \item kernel
      \item userspace
      \item hybrid
    \end{itemize}
    \pause
    \item green-threads and sparks
  \end{itemize}
\end{frame}

\begin{frame}{Threading (2)}
  \begin{itemize}
    \item context switch only at \textit{safe points}
    \item save as little state as possible
    \pause
    \item accurate GC $\rightarrow$ expand/shrink thread's stack on demand
    \item OS threads: stacks are immovable
    \item GHC's threads: 0-copy threads
  \end{itemize}
\end{frame}

\begin{frame}{Green-threads problems}
  \begin{itemize}
    \item blocking
    \item foreign calls (FFI or library)
    \item parallelism (run on all available cores)
    \item library constraints
    \begin{itemize}
      \item OpenGL, GUI libraries: API calls done on specific threads
      \item thread-local state
    \end{itemize}
  \end{itemize}
\end{frame}

\begin{frame}{Perseverance Paid}
  \begin{itemize}
    \item Haskell threads makes foreign call $\rightarrow$ OS thread for the
    other Haskell threads
    \item small pool of OS threads
    \item RTS multiplexing: transparent M:N threading model
    \pause
    \item \textit{capabilities}:
    \begin{itemize}
      \item \textit{nursery} -- access to new object creation
      \item each OS thread must hold these
      \item each one is owned by a single OS thread
    \end{itemize}
    \pause
    \item API for binding Haskell threads to OS threads
  \end{itemize}
\end{frame}

\begin{frame}{One Small Problem}
  \begin{itemize}[<+->]
    \item \textit{safe points} for context-switching
    \item memory allocation
    \item tight-loops -- no memory allocation
    \item frequent in benchmarks, less frequent in actual code
    \item no safe points $\rightarrow$ unable to schedule
    \item compromise between saving every cycle and having scheduling
  \end{itemize}
\end{frame}

\begin{frame}{Context Requiring Some New Manager}
  \begin{itemize}
    \item server applications
    \pause
    \item performance matters
    \item scalability
    \pause
    \item majority of applications -- HTTP servers
    \begin{itemize}
      \item most clients are idle most of the time
    \end{itemize}
  \end{itemize}
\end{frame}

\begin{frame}{Missing Things}
  \begin{itemize}
    \item support for a really large number of connections
    \item large number of active timeouts
  \end{itemize}
\end{frame}

\begin{frame}{Native Threads}
  \begin{center}\pgfuseimage{native}\end{center}
  \begin{itemize}
    \item \texttt{Apache}
    \pause
    \item clear code
    \item load balancing on cores
    \pause
    \item many context-switches
  \end{itemize}
\end{frame}

\begin{frame}{Event Driven}
  \begin{center}\pgfuseimage{event}\end{center}
  \begin{itemize}
    \item \texttt{Lighthttpd}
    \pause
    \item no process switching
    \pause
    \item single-core usage
    \item fragmented code
  \end{itemize}
\end{frame}

\begin{frame}{One Process per Core}
  \begin{center}\pgfuseimage{ppc}\end{center}
  \begin{itemize}
    \item \texttt{nginx}
    \pause
    \item core load-balancing
    \pause
    \item fragmented code
  \end{itemize}
\end{frame}

\begin{frame}{GHC's Green Threads}
  \begin{center}\pgfuseimage{green}\end{center}
  \begin{itemize}
    \item \texttt{warp}, \texttt{WAI}, \texttt{mighty}
  \end{itemize}
\end{frame}

\begin{frame}{Problems with \texttt{select}}
  \begin{itemize}[<+->]
    \item non-scalable
    \item O(number of connections)
    \item limited number of file descriptors
    \item must iterate on file descriptor set
  \end{itemize}
\end{frame}

\begin{frame}{A New Haskell IO Manager}
  \centering
  \pgfuseimage{mail}
\end{frame}

\begin{frame}{A Scalable IO Manager}
  \begin{itemize}
    \item \texttt{epoll}/\texttt{kqueue} event loop
    \item don't use lists for timeouts or other things
    \begin{itemize}[<+->]
      \item trees and heaps
      \item priority queues
      \item adjust/cancel pending timeouts
      \item data structures specialised based on contained type
      \item unpacked data types
    \end{itemize}
  \end{itemize}
\end{frame}

\begin{frame}{Network API}
  \begin{itemize}
    \item memory-mapped I/O
    \item zero-copy transfer (\texttt{sendfile})
    \item scatter-gather I/O (\texttt{writev})
    \item async socket I/O API (\texttt{TCP\_CORK})
  \end{itemize}
\end{frame}

\begin{frame}{Code Changes :: The Old Manager}
  \centering
  \pgfuseimage{old}
\end{frame}

\begin{frame}[label=new]{Code Changes :: The New Manager}
  \centering
  \pgfuseimage{new}
\end{frame}

\begin{frame}{Results :: Thread Timeout Delay}
  \centering
  \pgfuseimage{delay}
\end{frame}

\begin{frame}{Results :: Scalability}
  \centering
  \pgfuseimage{scalability}
\end{frame}

\begin{frame}{Results :: Sending 1M 1-byte Messages Through Pipes}
  \centering
  \pgfuseimage{pipes}
\end{frame}

\begin{frame}{Results :: Ping-Pong Benchmark}
  \centering
  \pgfuseimage{pong}
\end{frame}

\begin{frame}{Results :: \texttt{mighty} vs \texttt{nginx}}
  \centering
  \pgfuseimage{mightynginx}
\end{frame}

\begin{frame}{TODOs}
  \begin{itemize}
    \item Windows \pause -- \texttt{IO Completion Ports}
  \end{itemize}
\end{frame}

\begin{frame}{Resources}
  \begin{itemize}
    \item \href{http://learnyouahaskell.com/}{Learn You a Haskell}
    \item \href{http://book.realworldhaskell.org/read/}{Real World Haskell}
    \item \href{http://www.aosabook.org/en/ghc.html}{Architecture of Open Source Application. GHC Chapter}
    \item \href{http://hackage.haskell.org/trac/ghc/wiki/Commentary}{GHC Commentary}
    \item \href{https://github.com/ghc/ghc}{GHC's Source code}
  \end{itemize}
\end{frame}

\section{Extra slides}
\frame{\tableofcontents[currentsection]}

\begin{frame}{Scheduler}
  \begin{itemize}
    \item OS threads
    \begin{itemize}
      \item support non-blocking foreign threads
      \item support SMP parallelism
      \pause
      \item \texttt{+RTS -N} option
      \item more might be created
      \item kept in pool
      \item attached to capabilities
      \item platform-independent abstraction
    \end{itemize}
  \end{itemize}
\end{frame}

\begin{frame}{Scheduler (2)}
  \begin{itemize}
    \item green threads
    \begin{itemize}
      \item Thread State Objects
      \item garbage collected
      \item bound/unbound
      \item fields
      \begin{itemize}
        \item \texttt{link} -- list of TSOs
        \item \texttt{what\_next} -- what to resume
        \item \texttt{why\_blocked} -- why is thread blocked
        \item \texttt{bound}
        \item \texttt{cap} -- capability
        \item $\ldots$
      \end{itemize}
    \end{itemize}
  \end{itemize}
\end{frame}

\begin{frame}{Scheduler (3)}
  \begin{itemize}
    \item tasks
    \begin{itemize}
      \item abstraction over OS thread
      \item thread-local storage
      \item mutex and condition variable for OS threads synchronisation
      \item capability
    \end{itemize}
    \pause
    \item capabilities
    \begin{itemize}
      \item virtual CPU for executing Haskell code
      \item count should be close to the number of CPU cores
      \item \textit{thread holding a capability -- not blocked}
      \item spinlocks, deterministic locking
    \end{itemize}
    \pause
    \item sparks
    \begin{itemize}
      \item for \texttt{par} operator
      \item optimistic threading
    \end{itemize}
  \end{itemize}
\end{frame}

\begin{frame}{GHC's Concurrency RTS}
  \begin{itemize}
    \item give control to the programmer
    \item without decreasing performance or increasing complexity
    \item lift parts of RTS to Haskell library code
    \item concurrency substrate
  \end{itemize}
\end{frame}

\againframe{new}

\begin{frame}{Concurrency Substrate}
  \begin{itemize}
    \item minimal set of primitives
    \begin{itemize}
      \item create threads
      \item schedule them
      \item synchronization mechanism in multiprocessor context
    \end{itemize}
    \pause
    \item user-level concurrency libraries
  \end{itemize}
\end{frame}

\begin{frame}{PTM}
  \begin{itemize}
    \item primitive transactional memory
    \item lockless code
  \end{itemize}
  \input{code/ptm.hs}
\end{frame}

\end{document}
